\documentclass[11pt]{article}
\usepackage{graphicx, amssymb, amstext, amsmath, epstopdf, booktabs, verbatim, gensymb, geometry, appendix, csquotes, color, colortbl, hyperref} 
\usepackage[table]{xcolor}
\geometry{letterpaper}

\newcommand*\Title{Rebooting Compute Canada}
\newcommand*\cpiType{Building a \textbf{successful} and \textbf{federal} computational~research~enterprise,~together}
\newcommand*\Date{\today}
\newcommand*\Author{The Compute Canada Community}
\title{\Title}
\author{\Author}
\date{\Date}
%-----------------------------------------------------------

\usepackage{cpistuff/cpi} 

\begin{document}

\begin{titlepage}
\maketitle
\end{titlepage}

\linespread{1.15} 

\begin{executive}

\frame{%
\begin{description}
	\item[A New Federalism] Over time, the national
	Compute Canada project has gone from being a loose association
	of occasionally-cooperating independent sites to a
	highly centralized expensive central office making extremely
	specific technical and personnel prescriptions for the
	entire country.  Neither approach is sustainable.
    \begin{itemize}
        \item{\textbf{Standardization}}
        \item{\textbf{Facilitation}}
    \end{itemize}
	\item[Researcher Focus]  foo
    \item[Research Focus] bar
\end{description}

}
\end{executive}

\section*{Background}

Compute Canada has failed.  We need the next one to succeed.

Compute Canada --- the national project --- was assembled to
drastically improve the capabilities of Canadian research.
Investigators were too often limited in the scholarship they could
perform because of lack of local research computing expertise, or
availability of storage or computational resources.  The solution
to this problem was clear then, and is still clear now.  The most
scarce and valuable of our resources, the expertise of research
computing staff, actually grows, rather than is diminished, by
applying it to many diverse problems; and economies of scale apply
to computational and storage resources, which can be efficiently
shared. Thus, the best way to support Canadian research was to pool
resources nationally, rather than building many small silos. The
federal government agreed, providing a steady stream of federal
operating funding, far above what similar organizations have available
elsewhere, freeing our effort from relying solely on institutions
or even provinces.

But at this point it is clear that Compute Canada --- the actual existing
organization --- has failed.  Rather than building on the strengths
of the regional organizations and the national perspective of a
central group, the existing structure has radically worsened the natural tensions 
and distrust between them, escalating into a spiralling cycle of control and
dismissive centralization on one side, and increasing disengagement and resistance
on the other.

We thus find ourselves in a situation where Compute Canada has a central office
costing millions of dollars a year --- the equivalent of approximately
40 average-sized NSERC Discovery grants at a time when funding
council budgets are frozen --- with little direct benefit to
researchers.  The central office drives all technical decision
making with a one-size-fits-all approach, such as the sorry and
continuing saga of the multi-year \enquote{emergency} procurement
of storage.  But even with this level of centralization, there still
is very little meaningful cooperation or knowledge-sharing between
institutions; and Ph.D.-level computational scientists, rather than
being active research partners in projects suited to their expertise
regardless of their location across the country, largely spend their
time answering help-desk-level questions about compiler errors and
file system issues from users who happen to use the same computer
that they sit beside.  Lacking both the efficiency that comes from
centralization and the flexibility that comes from decentralization,
the community currently has the worst of both extremes.

A better future is possible, but the necessary changes will not
just happen on their own.  It is not in CFI's power or ability to
step in and impose a new vision of what a Canadian computational
research support enterprise will be.  Nor will the ministry insert
itself into the operations of a scientific support organization;
if it did, dozens of similar organizations across the country would
rightly be up in arms.

So if the situation is to be fixed, it is up to us --- the members, the regions,
and the researchers --- to do so.  The members and regions together do have
the power, and the duty to researchers, to reorganize Compute Canada and 
rebuild a more responsive national platform for computational research support.  

But we can't begin to fix our shared national computational research support 
enterprise without first knowing what \enquote{fixed} looks like.  An inwards-focused 
Operational Plan exercise was promised after the externally-focused 
strategic planning process in 2014; this never happened.  As a result, 
difficult but critically important internal questions were not debated and
never received consensus answers ---  What services do researchers need?
How do we best provide them?  What is the role of a central organization? 
What is the role of the regions?  How do decisions get made?

This document is one proposed set of answers to these questions; a sketch of
one possible alternate future for providing computational research support 
in Canada.  Its intention is to spur debate and inspire other different
sets of answers, starting a discussion that should have happened three years ago.
When a consensus emerges within our community, we will have both the knowledge
of where we want to go, and the power to get there.  Canadian research needs 
computing and data, and Canadian research deserves better than the current 
situation for providing them.

\section*{Principles}

Any structure for a Compute Canada, present or future, must be
judged against a set of principles we have for the running of such
a research support organization. 

We propose six such principles, listed below.  In this section we describe
them at length, and their current status.  

\begin{table}[ht]
\centering
\small {\sffamily
\rowcolors{2}{white}{gray!15}
\begin{tabular}{l|p{3.5in}}
\textcolor{cdaRed}{\textbf{Principle}} & \textcolor{cdaRed}{\textbf{Description}} \\
\hline
\hline
Researcher-Centred & The driver for every decision is researcher needs, with technology a means to an end. \\
Service Oriented & The purpose of the organization is enabling research, not just providing cycles. \\
Modern & Full advantage of current best practices are taken where they improve researcher experience. \\
National & Resource decisions are not made based on location - neither to researchers nor internally. \\
Interoperable, not Identical & All parts of the natural platform must interoperate seamlessly, but they should not be identical. \\
Equal Federal Partners & As equal funding partners, provinces and national operations share different but equally important responsibilities. \\
\hline
\end{tabular}
}
\end{table}

\subsection*{Researcher-Centred}

In any long-running organization, there is a tendency to lose the
perspective of the clients and instead to make decisions based on
what is easiest or best internally.  Groups that solve
problems using technology are doubly prone to this, as the technology
begins to seem important for its own sake, rather than simply being
a way to achieve success for a client.

Modern agile software development addresses this problem by having
desired \enquote{User stories} -- a new task a user would want to
perform -- drive software development, with a product owner in
charge of prioritizing the user stories so that they genuinely
reflect the needs of the clients.  While this is more meaningful
for software development than service delivery, the basic method
shows what one approach to keeping the clients needs firmly centred
in decision making looks like, and how seriously many large companies
take it.

In earlier incarnations of Compute Canada, a goal of maintaining a
system in the top twenty of the Top 500 list of large systems was
occasionally suggested.  This correctly was never made into a formal
priority, because it is simply not a legitimate goal of a research
support organization.  Specific technical benchmarks are end goals
of technical organizations, not research support organizations.
The end goals of the latter can only be to effectively support
particular projects and programmes of research.  Some of those
efforts may indeed end up requiring such a system, or access to
such a system, but that would simply be a means to achieving a true
goal of the organization.

The difference between an organization that is focussed on its
clients and one whose focus is internal is reflected quite starkly
their behaviours, in particular where time and money is spent.  A
researcher-focussed technical organization casts decision making
in terms of researcher needs and successes rather than technical
implementation details, deferring such details until the last
possible minute, and pushing such decisions as close to the researchers
as possible.  A researcher-centered technical organization would
never, as an example, begin the process of issuing RFPs for compute
systems by drawing up prescriptive, detailed discussions of
interconnects, processors, and core counts, but instead the metrics
would be described in terms of use cases, job mixes, waiting times,
and other researcher-facing metrics.

Similarly, a researcher-centred technical organization does not take urgent
researcher needs such as storage, and pre-impose specific technical
architectures upon the storage types before issuing RFPs to vendors;
researcher-facing metrics are used, and any feasible solution with
sufficiently good metrics and costs are quickly and efficiently 
procured.

A researcher-centred organization doesn't shift internal bookkeeping
burdens onto the researchers, such as having multi-page sign-up
forms requiring third-party authorization and several day waiting
periods before access (compare for instance XSEDE or Amazon).

A researcher-centred organization allocates funding based on clear
and concrete current or near-term researcher needs, and avoids 
spending large amounts of money on nebulous goals with no immediate
driving need, such as untested \enquote{Research Data Management} solutions.

\begin{table}[ht]
\centering
\small {\sffamily
\rowcolors{2}{white}{gray!15}
\begin{tabular}{p{2.5in}|p{2.5in}}
\textcolor{cdaRed}{\textbf{Not Researcher Centred}} & \textcolor{cdaRed}{\textbf{Researcher Centred}} \\
\hline
\hline
Users must fill out many elaborate forms & Easy sign-up, renewals \\
Technology drives decision-making & Researcher goals drive decision-making \\
RFPs specify architecture, interconnects, feeds and speeds & RFPs specify job mixes, researcher-facing metrics \\
Projects and collaborations are launched for their own sake & Projects and collaborations undertaken to meet specific, concrete, researcher needs.\\
Allocation of funding driven by various priorities & Allocation of funding addresses current and near-future researcher needs.\\
\hline
\end{tabular}
}
\end{table}

In a researcher-centred organization, significant decisions can
always be justified in terms of making it easier for specific
researchers to tackle concrete current or proposed projects of theirs,
and the amount of resources allocated to that decision are proportional
to those goals.

\subsection*{Service Oriented}

Keeping the researcher central to decision-making will not automatically
ensure that one is offering the most valuable services possible;
researchers will not necessarily know to ask for services that have
not been routinely provided in the past.  To ensure one is offering
a full range of valuable research-enabling services, one must
constantly try new offerings, but in a disciplined and researcher-centred
way.  

New services can be routinely and inexpensively trialled with
pilot projects, particularly when the services revolve around expertise
rather than hardware (and it is these expertise services which will
generally provide the greatest value-add for most researchers).  
This approach can only work, however, when it is paired with 
a commitment to ruthlessly prune services that provide little value before
incurring too much cost.  An area where this approach is taken successfully
is training and education efforts, led by the regions and with little central
involvement, where enrollment provides immediate feedback as to interest.

In a technology-focussed research computing organization, the main
research computing service offered tends to be helpdesk-style
questions about logging in, compiler errors, or queuing jobs 
--- literally the lowest-level, least-value-added services that could
be meaningfully offered beyond having the systems running.
Compute Canada currently has approximately 60 Ph.D.-level staff who
spend much of their work time performing this level of support.

Other organizations elsewhere offer a much richer set of services.
Both XSEDE with their extended collaborative support
services\footnote{\url{https://www.xsede.org/ecss}} and the growing
Research Software Engineering\footnote{\url{http://rse.ac.uk}} role
in the UK embed staff inside research groups for extended periods
of time to provide a variety of expertise, which can be particularly
valuable for groups new to computational research or trying new-to-them
approaches.  Such staff participate deeply in the research, often
to the level of authorship, and manifestly enable research that
would have happened more slowly or not at all otherwise.  In the
2013 staff survey, this level of participation was mentioned often
as a stated wish of the regions' trained and ambitious technical
staff.  

SHARCNET has long offered dedicated programmer time, one type of
such services, and it has been quite successful and indeed very
popular with both researchers and staff.  Such efforts have not yet
been trialled nationally.

While Compute Canada is staffing on programmer staff currently,
significantly, such staff are being hired centrally, report solely
to senior executives, and skill sets have been selected and staff
hired without any discussion with the researchers that they are
proposed to enable.  The intention appears to be for middleware
development, so that such staff have little research computing
background to speak of, but it is only now after most of the hiring
has been done that assessment of what middleware is needed by the
RPP and CyberInfrastructure projects they aim to support is being
done.

\begin{table}[ht]
\centering
\small {\sffamily
\rowcolors{2}{white}{gray!15}
\begin{tabular}{p{2.5in}|p{2.5in}}
\textcolor{cdaRed}{\textbf{Not Service Oriented}} & \textcolor{cdaRed}{\textbf{Service Oriented}} \\
\hline
\hline
New services are chosen centrally and rolled out full-scale nationally. & New services are piloted, tested, and scaled-up or phased out.\\
Services tend to be low-level and low-value-add. & Services range from hardware-provision to research partnership.\\
Services are either devised centrally, or done \enquote{the way things have always been done} & Best practices and new services used successfully elsewhere are routinely trialled. \\
\hline
\end{tabular}
}
\end{table}

A service-oriented research support organization ensures that services are offered
to enable research at all stages of a project and at all levels of involvement,
taking full advantage of expertise and resources available to the organization.
New services are rigorously trialled with pilots before rolling out nationally, 
and service offerings are pruned if unnecessary.

\subsection*{Modern}

A research service organization which uses technology to address
researcher needs must stay on top of new tools so that they can
meet those needs as effectively as possible.  Researcher needs must
always be the driver, but solutions change quickly.

Those tools can certainly be new hardware --- NVMe, FPGAs, and
server-class ARM CPUs are all technologies which could have significant
impact on research computing in the quite-near future --- but they
can also be new technologies for robustly and efficiently providing
technical services.

As more and more companies rely on computer infrastructure, the
past decade and a half have led to improved approaches to ensuring
the services they provide are reliable and effective.  For instance,
servers and network connections fail; rather than being blindsided
by the predictable, Netflix took the approach of routinely and
automatically testing of failure to ensure that individual failures
did not adversely effect users.  Similarly, Google pioneered a now
widely-adopted Systems Reliability Engineer (SRE)
approach\footnote{\url{https://landing.google.com/sre/book.html}}
which emphasizes extensive automation, minimizing human intervention
on routine operations (even failures), allowing staff to focus on
providing better kinds of services.

An organization which adopts modern tools ensures there is paid
staff time and training for learning about new hardware and new
approaches to deploying them.  It continually provides small
experimental systems to the staff (and interested researchers) to
explore the suitability of new hardware and new provision techniques
for suitability of research systems.  It tests, modifies, and deploys
new approaches to systems management.
It also takes seriously the possibility of using commercial
cloud providers as service provision options for some use cases.

\begin{table}[ht]
\centering
\small {\sffamily
\rowcolors{2}{white}{gray!15}
\begin{tabular}{p{2.5in}|p{2.5in}}
\textcolor{cdaRed}{\textbf{Does Not Use Modern Tools}} & \textcolor{cdaRed}{\textbf{Uses Modern Tools}} \\
\hline
\hline
No availability of experimental systems & Invests in new technology for staff to explore for suitability for researcher use \\
Little paid staff training & Provides staff with time and training in new methods and techniques \\
Runs computer systems as they were run in late '90s & Modern SRE approaches are explored, customized, and used, such as heavy automation, routine failure testing \\
Interaction with users same as in late '90s & Interaction with researchers followed using tools like CRMs, so new staff anywhere in country can quickly come up to speed\\
Commercial cloud providers are the competition & Commercial cloud providers are one of many provision options \\
\hline
\end{tabular}
}
\end{table}

An research support organization which adopts modern tools also takes advantage of
tools used elsewhere to provide better services, such as following researcher interactions
and project progress using tools like Customer Relationship Management (CRM) packages, 
so that staff anywhere in Canada who might be able to bring their expertise to bear to
assist the researcher can quickly be brought up to speed.

\subsection*{National}

Any conversation about Compute Canada has to have as a starting
point that the reason for the effort is that pooling resources
nationally is the best way to support Canadian researchers, and
that their location in the country cannot matter for the type and
level of services receive.

Truly national provision of resources to researchers, particularly
resources as diverse and important as expertise, is something which
takes active effort on the part of the research support organization;
it can't be neglected as something which is allowed in principle but
left to the researcher to pursue on their own.  Presenting researchers
with a list of national staff and bullets list of their expertise,
and leaving the researcher to try contacting staff members in turn to
recruit them to collaborate in their project, is a woefully inadequate
approach to enabling computational research projects.

A truly national organization must make sure that Canadian researchers
in all fields and institution types are adequately supported.
Researchers in biological and life sciences (particularly human
health), social sciences, and scholars in the digital humanities
remain poorly served by Compute Canada; applied research work in
colleges and polytechnics (over \$200M/yr of external funding, approximately
40\% of which comes from the private sector) is essentially completely ignored.

A truly national research support organization can't revert to
using funding formulas that divvy funding up by the number of
users in geographical catchment areas, but must fund services and
providers to support researchers nationally.

\begin{table}[ht]
\centering
\small {\sffamily
\rowcolors{2}{white}{gray!15}
\begin{tabular}{p{2.5in}|p{2.5in}}
\textcolor{cdaRed}{\textbf{Not Truly National}} & \textcolor{cdaRed}{\textbf{Truly National}} \\
\hline
\hline
Researchers are given a list of national resources available for them to investigate themselves & National teams of resources are actively assembled for a project  \\
Researchers in some fields or institution types are overlooked & Researchers are supported equally across the country, across all institution types \\
Providers and services are funded based on the number of users near their location & Providers and services are funded based on the value they provide to the national community \\
\hline
\end{tabular}
}
\end{table}

\subsection*{Interoperable, not Identical}

The internet is arguably the most important computational tool for
enabling faster and better research made in modern times, and yet the central
internet body, the IETF, does not specify brands of computer and
browser, nor enforce a list of services that every website must
provide.  Instead, strict interoperability requirements, coupled
with the freedom to innovate within those standards, have combined
to make the internet such a powerful research tool.

Similarly, the Global Alliance for Genomics and Health
(GA4GH\footnote{\url{http://genomicsandhealth.org}}) is an international
effort to build computational research tools to make full use of
the increasing volume of genomics data to improve human health.
A recurring mantra of the effort is to \enquote{Co-operate on interfaces,
compete on implementations}.  By building interoperability standards,
allowing specialization for implementations, and working iteratively,
the project is enabling new efforts like the Beacon~Network\footnote{\url{https://beacon-network.org/}}
for data discovery and the Matchmaker~Exchange\footnote{\url{http://www.matchmakerexchange.org}}
for better understanding rare diseases.

The Canadian research environment can be strengthened by ensuring
that each project has the potential to access the complete national
portfolio of computational science resources.  But specifying exact
model numbers of hard~drives to use nationally, or that each region
provide copies of the same services to the national research
community, profoundly misunderstands the point of working together
and pooling resources.

Onboarding

\begin{table}[ht]
\centering
\small {\sffamily
\rowcolors{2}{white}{gray!15}
\begin{tabular}{p{2.5in}|p{2.5in}}
\textcolor{cdaRed}{\textbf{Focused on Identical}} & \textcolor{cdaRed}{\textbf{Focused on Interoperable}} \\
\hline
\hline
Infrastructure is specified in terms of technical specifications & Infrastructure requirements specified in terms of SLAs and interfaces to other infrastructure \\
Services are replicated across the country & Sites and regions can specialize in service provision with clear interoperability specifications \\
New sites cannot fully join the platform without wholesale replacement of infrastructure, procedures & New sites can easily fully join the platform by exposing services, infrastructure via interfaces \\
\hline
\end{tabular}
}
\end{table}

\subsection*{Equal Federal Partners}

Canada has one of the most fiscally decentralized federal governments
in in the G20, particularly when it comes to funding of research.  This
flexibility has real benefits, but it introduces complexities that
are just as real.  It is for this reason that we can not simply
copy successful organizational models from the UK, or from XSEDE
(from the US, where states generally play very little role in funding
research). Even the EU is of little help to us here; their pan-Europe
effort, PRACE, focuses exclusively on one type (\enquote{Tier 0})
of research computing, with all other aspects of research computing
support expected to be supplied to researchers by their member
states or institutions.


\section*{Rebuilding To Our Principles}

Having laid out a set of guiding principles, one can begin to design
a new national organizational structure that adheres to them, aligning
better to the expectations of the community.

The main building blocks of our national project include a national
office; the provincial or regional organizations (henceforth, the
regions), which we take to include the sites and the institutions
that host them.\footnote{The relationships between the institutions and
the regional organizations are important and complex, but they will quite
rightly vary from province to province; since it's not meaningful 
to have a national consensus on the nature of those relationships,
they aren't discussed here.}  The two have different relative strengths.


\subsection*{Role of a National Office}

\subsection*{Role of the Provinces and Regions}

\subsection*{Decision Making}

\subsection*{Budgeting}

\subsubsection*{Capital}

\subsubsection*{Operations}

\section*{What's Next}

Our shared national project of enabling Canadian research with a
country-wide portfolio of resources and expertise is too important
to do poorly; and it is too important for us to retreat back into
silos and limit researchers to those experts and computers that
happen to be nearby.

After having read this document, you likely have something to say.
Hopefully there were parts of this proposal that you strongly agree
with; even better, there are probably parts you disagree with, think
are missing, or think should go missing.

The purpose of this document is not to advocate in particular for
the proposals contained within (although the points made here are
genuinely-held, not just straw-man arguments). The purpose is to
start in earnest a conversation that should have been launched
officially three years ago, allowing the community and stakeholders
to come to a consensus about what the internal organizational
structure of Compute Canada should be, how it should make decisions,
and how it should offer services to Canadian researchers and scholars.

The most important next step, then, is for you to have this discussion
with colleagues locally and across the country, disagreeing vehemently
initially on some points, and coming to agreement on others.  We
have put together one forum to have such discussions at
\url{https://www.rebootcompute.ca}, where we would also be delighted
to host competing proposals, but the location of the discussions
doesn't matter; that they take place does.

The members and regions can rebuild a Compute Canada that works,
and works the way the community wants it to, but they cannot do so
without knowing what destination the community feels they should
aim for. As the title of this document suggests, getting there from
here will require completely turning off Compute Canada before
starting it up again, with a completely new board and staff that
are completely committed to the model and priorities that the
community have chosen.  But this process can happen in months, not
years, and the result will be a Canadian research community served
by a successful, truly federal, national computational research
support organization.  The Canadian research and research computing
communities can do great things together.  Let's get started.

\end{document}
