\documentclass[11pt, letterpaper, twoside]{article}
\usepackage{graphicx, amssymb, amstext, amsmath, epstopdf, booktabs, verbatim, gensymb, appendix, csquotes, color, colortbl, hyperref}
\usepackage[table]{xcolor}
\usepackage[most]{tcolorbox}
\usepackage{natbib}

\newcommand*\Title{Rebooting Compute Canada}
\newcommand*\cpiType{Building a \textbf{successful} and \textbf{federated} computational~research~enterprise,~together}
\newcommand*\Date{\href{http://www.rebootcompute.ca}{www.rebootcompute.ca}}
\newcommand*\Author{\parbox{2.5in}{A Discussion Paper For\\ The Compute Canada Community\\by Jonathan Dursi\\and Jill Kowalchuck}}
\title{\Title}
\author{\Author}
\date{\Date}
%-----------------------------------------------------------

\usepackage{cpistuff/cpi}

\begin{document}
\pagenumbering{roman}

\begin{titlepage} \maketitle \end{titlepage}

\cleardoublepage

%\linespread{1.15}

\thispagestyle{empty} 

\begin{executive}

Computing and data, and the expertise and tools to make use of both, is
now central to all fields of study. Ten years after the creation of
Compute Canada in response to the National Platforms Fund call, and
after the Naylor Report on science funding, it is an apt time for the
Canadian community built around this national research platform to take
stock. Is it doing what we need it to do for Canadian researchers? Is it
working the way we want it to? What should a Canadian computation and
data platform for supporting research look like in the coming years?
This document aims to begin that discussion within the community.

Here we propose seven principles to guide us in this discussion --- that
our project should serve Canadian research in a researcher-centred,
service-oriented, and truly national way; and that it should operate as
a true federation of equal partners, interoperable but not identical,
collaborative and up-to-date. We suggest in particular that it is vital
that our national platform is adaptive and responsive to researchers,
making choices driven by research needs and not technical choices, and
should make full use of the diversity and specialization that a Canadian
federation and its partners offer.

From those principles, we make evidence-based proposals for a renewed
Canadian organization. Comparisons with successful examples of federated
organizations within Canada and abroad suggest that while the basic
architecture of our federation is sound, important roles and
relationships need to be clarified. While a central office must be
responsible for the processes of defining priorities, strategies, and
standards of interoperability, a successful federation requires those
processes to have buy-in from partners committed to the goals of the
federation. The Board of Directors of the central office in a federation
must have experience and training to handle the delicate task of
governing a central office but being responsible to a national
community. The Members need adequate visibility into the operations of
the central office and the federation as a whole so that they can
support their vital role to the organization. And that engagement needs
to extend to all who are invested in the success of research in Canada:
regional staff and Boards, institutional staff, researchers and funders,
and other organizations that provide digital infrastructure for research
in Canada. This document focusses on Compute Canada in particular, but
the principles and proposals apply to any digital research
infrastructure providers, or the system as a whole.

Success for this document will mean starting conversations, inspiring
other documents and differing points of view, and the emerging of a
consensus within the community of what a renewed national platform for
the next ten years looks like. That does not mean this document is a
straw-man. The authors have played roles in the national platform
starting at its inception, from researcher to consortium and regional
(east and west) staff and management, and within the Compute Canada
central office, and hope that experience plus the benefit of some
distance have produced a coherent and compelling vision of what the
Compute Canada national project could be. But what matters is not this
proposal; it is what the community as a whole decides it wants its
national platform to be.
\end{executive} 

\newpage
\listofproposal
\newpage
\tableofcontents

\cleardoublepage
\pagenumbering{arabic}

\section*{Introduction} 
\addcontentsline{toc}{section}{Introduction}%

Fundamental to supporting research in Canada are a handful of questions:
How do we know what services researchers need? How do we best provide
them? What is the role of a central office? What are the roles of
provincial organizations, and institutions? How do decisions get made?

This document contains one proposed set of answers to these questions
for the Compute Canada national platform; a sketch of one possible
alternate future for providing computational research support in Canada.
Its intention is to begin a discussion within the community, where we
look forward together and ask how computing- and data-powered research
should be supported.

The Compute Canada national project for supporting research through
computing and data was assembled over the past ten years to drastically
improve the capabilities of Canadian research. Investigators were too
often limited in the scholarship they could perform because of lack of
local research computing expertise, or availability of storage or
computational resources. The solution to this problem was clear then, as
it is now.  The scarcest and most valuable of our resources, the
expertise of research computing staff, grows rather than being
diminished by applying it to many diverse problems; economies of scale
apply to the sharing of large computational and storage resources. Thus,
the best way to support Canadian research is to provide and share
resources, rather than building small silos.

No one can deny that the research that has been enabled by the hardware
and expertise provided over the years is, and continues to be,
exceptional. However, like any organization it must strive to be better
and challenges must be addressed with input from the entire community.
Over time, the project named Compute Canada has gone from being a loose
association of occasionally-cooperating independent sites to an
organization with a sizeable central office where a great deal of
decision making is consolidated. In this document, we argue that neither
approach is sustainable; neither provides the best results for Canadian
research; neither is ambitious enough in what we can do for our
researchers and scholars. We instead present a vision for Compute Canada
for the next ten years that is national, but not centralized; that is
diverse, but interoperable; and that is focussed on supporting Canadian
research with a wide range of services reflecting the breadth and depth
of Canadian scholarship.

Computation and data plays a central role in all fields of study; the
national project should aim for nothing less than to give Canadian
researchers an unfair advantage in tackling problems that are important
to Canada. Suggestions have been made by Canada's Fundamental Science
Review (the Naylor report), and will be made by an upcoming Leadership
Council for Digital Research Infrastructure report; but now is the time
for we the community including researchers, staff, the members and all
who support Canadian research to take a step back and have a real
discussion about where to go to next. It is time for us to decide what
we want computational support for Canadian research to look like, and
how it should work.

Any Compute Canada, present or future, must be judged against a set of
principles to guide the research support organization.  We propose seven
such principles: that a Canadian computation and data research support
platform should serve Canadian research in a researcher-centred,
service-oriented, and national way; and it should operate as a true
federation, interoperable but not identical, collaborative, and
up-to-date. From those principles, and based on the experiences of other
federated organizations in Canada and abroad, we suggest concrete
organizational improvements that could help us move towards our goals.

For motivation and concreteness, we begin with an attempt to illustrate
what such a future federation would look like, from the most important
point of view --- that of the researcher.

\section*{Prologue: The New Project}
\addcontentsline{toc}{section}{Prologue: The New Project}%

\begin{tcolorbox}[enhanced,breakable,colback=gray!15,colframe=cdaRed,parbox=false]
\sffamily \textbf{May 2, 2022} The paper was coming together nicely,
thought Dr. Shannon Banks, a postdoc at the University of Western
Manitoba. She had a meeting with her PI later in the week, and there
was a pretty good chance she could have a mostly-complete first draft
ready by then.

It was a little surprising how quickly the work had gone. She had joined
Prof. Reeve's group bringing experimental and data-analysis expertise,
but extending the work through comparison to simulations had required
developing new skills and she couldn't lean on her new colleagues for
much help: as someone who had analyzed quite-large experimental datasets
during grad school abroad, she \textit{was} the local computational
expert in the growing experimental group.

When she had signed up for her National Platform account, a quick
process requiring just an institutional email address, a consultant
analyst named Walter Payne introduced himself. He was at a nearby
university, and had experience in a related field. While help desk staff
had handled her inevitable login, compiler, and queue questions, as well
as walking her and Prof. Reeve through the process of getting a starter
allocation, Walter would be there to make sure more ``science-y''
questions from the Reeve group got seen by the right person.

After Shannon explained the simulations she wanted to run and sent along
pointers to relevant papers, Walter ended up introducing her to Stella
Gregory, another consulting analyst stationed in Nova Scotia; Stella had
run very similar sorts of simulations herself, and suggested a slightly
different approach which would give significantly better results. It
took a couple of video calls, but afterwards Shannon had a pretty good
idea how to proceed, especially with the online interactive training
material Stella pointed her to.

When the simulations with varying parameters finally started producing
results (quite quickly; the systems seemed much more flexible and stable
than those she was used to), Shannon visualized the first few results on
her laptop --- but then the results kept coming!  The tutorials had
described how to automate the visualizations and what to do with her
data. There was another cluster elsewhere in the platform (in BC? It
wasn't clear, but apparently BC was big into visualization) that had
GPUs and a file system better suited for this sort of work. The same
tools that helped her manage her data to ensure she met her funders data
sharing requirements allowed her to move the data, so migrating the
files to and from was pretty painless. In fact, it went so smoothly that
the one time that it hung for ten minutes, she wrote an email to the
helpdesk wondering if she had done something wrong. By the time she hit
Send, it had started up again.  Helpdesk proudly, if somewhat
cryptically, explained that a transfer server had crashed, but that they
routinely tested all sorts of failures (with something or someone called
a mischief monkey) and such things almost got fixed or restarted
automatically within a half-hour or so.

When the visualizations were far enough along to be able to make movies,
the team at the weekly group meeting was so impressed that she almost
felt a little sheepish. Or, at least right up until the new grad student
asked if she could now analyze the simulated data using exactly the same
pipeline as the experimental data, and compare it with two other data
sets she had found from researchers in Ontario and Australia; Prof.
Reeves started enthusing about the idea.

Shannon nodded and mumbled something noncommittally optimistic, but left
the meeting frowning --- there was \textit{much} more simulation output
than experimental data and at \textit{way} higher resolution; it wasn't
at all clear that this was going to work. There had been a lot of
experimental data, sure, but it was mainly about working with large
numbers of small datasets. Dealing with this much higher resolution data
might be an issue. When she tested the Python scripts on her desktop,
they crashed almost immediately.

She contacted Walter, describing her algorithm, the distributed data
sets and the problem. He said that he'd look into the issue further and
pointed her to to some upcoming online Python parallel programming
courses.

A couple of days later she was contacted by Carolyn Malone, who
explained she was a performance specialist leading the team working on a
prototype nonvolatile memory system in Qu\'ebec. They were trying out
this system to see how useful it would be for certain kinds of data
analytics. Walter had raised the possibility of Shannon's project as a
test case; Carolyn had gone through the research project relationship
management system reading up on what Shannon had been doing (ah, so
\textit{that's} why Walter always had her contact him through the ticket
manager rather than directly!) and had seen that this was a classic
application for an old-is-new-again external memory algorithm. If
Shannon was willing to use a small and slightly flaky test new system,
Carolyn could have Andy Bell, an analyst on the team, help her with
coding it up --- it would be a week or so of effort, and a new use case
for the pilot team, and it should (no promises) get the results she
needs.

Shannon agreed, and two weeks later was happy to show her group,
including the ``helpful'' new grad student, the results. And now the
manuscript was almost ready for group feedback, with Andy and Stella on
the author list, and acknowledgements to Carolyn, and of course Walter.
\end{tcolorbox}

\section*{Principles}
\stepcounter{section}
\addcontentsline{toc}{section}{Principles}%
\addcontentsline{pro}{section}{Principles}%

We propose the following seven principles to guide our discussion 
of what our national platform could be.

\begin{table}[ht]
\centering
\small {\sffamily
\rowcolors{2}{white}{gray!15}
\begin{tabular}{l|P{3.5in}}
\textcolor{cdaRed}{\textbf{Principle}} & \textcolor{cdaRed}{\textbf{Description}} \\
\hline
\hline
Researcher-Centred & The driver for every decision is researcher needs, with technology a means to an end. \\
Service Oriented & The organization aims to enable research in a number of well-defined ways. \\
National & The platform aims to support researchers as best possible, regardless of where the researchers or the resources are. \\
Equal Federal Partners & As equal funding partners, provinces, institutions, and the national office share different but equally important responsibilities. \\
Interoperable, not Identical & All parts of the national platform must interoperate seamlessly, but they need not and should not be identical. \\
Collaborative & All parties that support the platform are coming to the table in good faith to achieve a common goal. \\
Up-To-Date & Tools are offered where they improve the services and support for researchers. \\
\hline
\end{tabular}
}
\end{table}

\subsection*{Researcher-Centred}
\addcontentsline{toc}{subsection}{Researcher-Centred}%

\proposal{Research, and concrete researcher needs, should be
the basis for all decision making.}

All established organizations face the danger of losing the perspective
of those it serves. It becomes a little too natural to make decisions
based on what is easiest or best internally; this is especially true if
decisions are made several levels removed from those working directly
with the clients. Organizations that solve problems using technology are
doubly prone to this, as the technology begins to seem important for its
own sake, rather than simply being a way to help a client achieve
success. Note that in the Prologue, everything is arranged around
success for the researcher.

\begin{table}[ht]
\centering
\small {\sffamily
\rowcolors{2}{white}{gray!15}
\begin{tabular}{P{2.5in}|P{2.5in}}
\textcolor{cdaRed}{\textbf{Not Researcher Centred}} & \textcolor{cdaRed}{\textbf{Researcher Centred}} \\
\hline \hline
Users must fill out many elaborate forms & Easy sign-up, renewals, resource allocations\\
Technology drives decision-making & Researcher goals drive decision-making\\
RFPs specify architecture, interconnects, feeds and speeds & RFPs specify job mixes, wait times, researcher-facing metrics \\
Projects and collaborations are launched for their own sake & Projects and collaborations undertaken to meet specific, concrete, researcher needs\\ 
The researchers adapt to the way things are done & The way things are done adapt to the researchers\\
Researchers cobble together services across digital infrastructure providers & Digital infrastructure providers work closely together to provide seamless services researchers need \\
\hline
\end{tabular}
}
\end{table}

The difference between an organization that is focussed on its clients
and one whose focus is internal is reflected in behaviour, in particular
where time and money is spent. In a researcher-focussed technical
organization, the first question is always ``how does this help the
researchers''; it casts decision making in terms of researcher needs and
successes rather than technical implementation details, deferring such
details until the last possible minute, and pushing such decisions as
close to the researchers as possible.  Significant decisions can always
be justified in terms of making it easier for specific researchers to
tackle concrete current or proposed projects, and the amount of
resources allocated to that decision are proportional to those goals.

A researcher-centred organization must also ensure that they work
closely with other partners, so that researcher needs requiring
cooperation between service providers are met.

\subsection*{Service Oriented}
\addcontentsline{toc}{subsection}{Service Oriented}%

Keeping the researcher central to decision-making will not automatically
ensure that one is offering the most valuable services possible;
researchers will not necessarily know to ask for services that have not
been routinely provided in the past. One must constantly try new
offerings, but in a disciplined and researcher-centred way.

\proposal{A broad range of research-support services should
be offered, with new services continually piloted.}

New services can be routinely and inexpensively trialled with pilot
projects, whether they centre on providing expertise, hardware,
software, or a combination. The training efforts, currently led by the
regions, demonstrates the advantage of this approach. Enrollment
provides immediate feedback on demand and content allowing for nimble
program development.

\begin{table}[ht]
\centering
\small {\sffamily
\rowcolors{2}{white}{gray!15}
\begin{tabular}{P{2.5in}|P{2.5in}}
\textcolor{cdaRed}{\textbf{Not Service Oriented}} & \textcolor{cdaRed}{\textbf{Service Oriented}} \\
\hline \hline 
New services are chosen centrally and rolled out on a full scale nationally & New services are piloted, tested, and scaled-up or phased out\\
Services tend to be low-level and with less value-add & Services range from hardware-provision to research partnership\\
Services are either devised centrally, or done \enquote{the way things have always been done} & Best practices and new services used successfully elsewhere are routinely trialled \\
\hline
\end{tabular}
}
\end{table}

In a technology-focussed research computing organization, the main
research computing service offered tends to be helpdesk-style questions
about logging in, compiler errors, or queuing jobs.  Compute Canada
currently has approximately 60 Ph.D.-level staff and 30 with other
advanced degrees; it is critical that the federation makes as much use
of this skill and expertise to provide researchers the most important
added support, and retains these experts by providing meaningful
opportunities to contribute to research.

\proposal{Services offered elsewhere, such as having staff participate more closely in research, should be investigated.}

We can look to a variety of international organizations for examples of
successful service offerings. Examples include XSEDE's extended
collaborative support services\footnote{\url{https://www.xsede.org/ecss}}, and the growing
number of Research Software Engineers\footnote{\url{http://rse.ac.uk/}} in the UK.  Such
staff participate in the research, often to the level of authorship, and
manifestly enable research that would have happened more slowly or not
at all. In the 2013 Compute Canada survey of institutional and regional
staff, this level of participation was mentioned often as a desire
technical experts. In the Prologue, staff play several well-defined
roles in Shannon's project.

SHARCNET has long offered dedicated programmer time, one type of such
services, and it has been quite successful and indeed very popular with
both researchers and staff. Such efforts have not yet been trialled
nationally.

\subsection*{National}
\addcontentsline{toc}{subsection}{National}%

Any conversation about Compute Canada must have as a starting point that
Canadian researchers merit having access to a national portfolio of
resources, and that their location in the country cannot matter for the
type and level of services received.

\proposal{The platform must be available to the entire
Canadian research community, with specific efforts to efficiently
assemble the most appropriate resources to support new and existing
communities.}

Truly national provision of resources to researchers, particularly
resources as diverse and important as expertise, is something which
takes active effort on the part of the research support organization; it
can't be neglected as something which is allowed in principle but left
to the researcher to pursue on their own. Presenting researchers with a
list of national staff and bullet lists of their expertise, and leaving
the researcher to try contacting staff members in turn to recruit them
to collaborate in their project, is a woefully inadequate approach to
enabling computational research projects. In the Prologue, national and
diverse resources are actively assembled to enable Shannon's research.

\begin{table}[ht]
\centering
\small {\sffamily
\rowcolors{2}{white}{gray!15}
\begin{tabular}{P{2.5in}|P{2.5in}}
\textcolor{cdaRed}{\textbf{Not Truly National}} & \textcolor{cdaRed}{\textbf{Truly National}} \\
\hline \hline
Researchers are given a list of national resources available for them to investigate themselves & National teams of resources are actively assembled for a project  \\
Researchers in some fields or institution types are overlooked & Researchers are supported equally across the country, across all institution types \\
Services are replicated many times for provision to local users & Providers are encouraged to specialize to meet local priorities and needs while providing services to all \\
\hline
\end{tabular}
}
\end{table}


A truly national organization must ensure that Canadian researchers in
all fields and institution types are adequately supported.  Researchers
in biological and life sciences (particularly human health), social
sciences, and scholars in the digital humanities must be served as
capably as those in physics and biochemistry; effort must be taken to
reach out to applied research work in colleges and polytechnics (over
\$200M/yr of external funding, approximately 40\% of which comes from
the private sector).

Currently, computing resources for the very largest users of resources
are provisioned truly nationally, via the RAC process.

\subsection*{Equal Federated Partners}
\addcontentsline{toc}{subsection}{Equal Federated Partners}%

Canada has one of the most fiscally decentralized governments in the
G20. This flexibility has real benefits, but it introduces complexities
that are just as real, and is why there are no ready-made organizational
models for research support from abroad for us to copy for our national
project.

\proposal{The structure of our federation partnership must
reflect the reality of several funding partners. }

The majority of funding for Compute Canada is driven by the provinces
and institutions with only 40\% coming from federal sources.  The
provinces will reasonably have different priorities than the federal
government, and their priorities and existing capabilities will differ
amongst themselves. Any organizational structure or process that doesn't
acknowledge and accommodate these perfectly valid and healthy tensions
between equal funding partners will be too brittle to last.

\begin{table}[ht]
\centering
\small {\sffamily
\rowcolors{2}{white}{gray!15}
\begin{tabular}{P{2.5in}|P{2.5in}}
\textcolor{cdaRed}{\textbf{Unequal Federal Partners}} & \textcolor{cdaRed}{\textbf{Equal Federal Partners}} \\
\hline \hline
National office makes all decisions & National and provincial partners make decisions by consensus\\
National government gives money to provinces to spend however they want & Investments are made to build a country-wide platform that supports all researchers, with regional contributions that reflect regional priorities \\
Understanding of researcher needs limited to either \enquote{the researchers we've worked with} or \enquote{researchers in general} & Researcher needs both local and national, that have been supported already and not, are taken into account \\
\hline
\end{tabular}
}
\end{table}

The crass-but-practical concern of funding is an immediately clear
justification for this principle, but not the most important. Being
researcher-centred means taking all perspectives on researcher needs
into account, and the partners in federation have important but
different perspectives.

As the front-line service-providers to researchers, the regions have
immediate and hands-on experience knowing what the investigators they
are working with need. The central office, communicating directly with
national societies and funding agencies, and conducting needs
assessments, knows what researchers collectively need, and what is
currently lacking in the research ecosystem.

An effort to be researcher-centred based on only one of those
perspectives cannot succeed. A project undertaken with a general intent
to support researchers in the abstract can only end badly. And a project
undertaken to help those researchers that are already being helped, but
more so, will leave an ever-larger number of investigators behind.

Incorporating both perspectives equally is genuinely difficult. As
Canadians have known for 150 years, decision-making between federal and
provincial bodies can be a slow and sometimes frustrating process; but
the results are robust and durable, and are better decisions for having
had the multiple inputs. A platform that values the inherently federated
nature of our partnership, and interoperability rather than uniformity,
can build on the strengths and priorities of its participants rather
than trying to paper them over.

\subsection*{Interoperable, not Identical}
\addcontentsline{toc}{subsection}{Interoperable, not Identical}%

The internet is arguably the most important computational tool for
enabling faster and better research made in modern times, and yet the
central internet technical body, the Internet Engineering Task Force
(IETF), does not specify brands of computer and browser, nor does it
enforce a list of services that every website must provide each user.
Instead, strict interoperability requirements, coupled with the freedom
to innovate within those standards, have combined to make the internet
such a powerful research tool.

\proposal{The services offered by the national platform must
be interoperable, not merely identical.}

The Canadian research environment can be strengthened by ensuring that
each project is able to access the complete national portfolio of
computational science resources. But to focus on implementation details
rather than interoperability standards is to miss out on many of the
opportunities that come from that working together and pooling
resources. In the Prologue, Shannon interacts with several hardware
systems and people in varying regions, so that interoperability is
vital; implementation details are not.  Currently some of the national
teams, such as the security team, work under this model, defining
standards and best practices without specifying implementation details.

Focusing on interoperability rather than implementations allows
specialization, with different providers providing solutions tailored to
different use-cases; it allows experimentation, testing out new
implementations at one site without disrupting the platform as a whole;
it allows rapid prototyping and piloting of new approaches without
having to roll out homogeneous and potentially untested changes to the
entire country.

\begin{table}[ht] \centering \small {\sffamily
\rowcolors{2}{white}{gray!15} \begin{tabular}{P{2.5in}|P{2.5in}}
\textcolor{cdaRed}{\textbf{Focused on Identical}} & \textcolor{cdaRed}{\textbf{Focused on Interoperable}} \\
\hline
\hline 
Infrastructure is specified in terms of technical specifications & Infrastructure requirements specified in terms of SLAs and interfaces to other infrastructure \\
Experimentation requires lock-step changes across the country & Experimentation can be performed easily and locally \\
New sites cannot fully join the platform without wholesale replacement of infrastructure, procedures & New sites can easily fully join the platform by exposing services, infrastructure via interfaces \\ 
Little thought given to interaction with other digital infrastructure providers & Close collaboration and interoperation with other digital infrastructure providers \\
\hline 
\end{tabular}
}
\end{table}


Well-defined interoperability requirements also makes bringing new
providers into the platform easier. As opposed to requiring a new site,
already providing services, to completely change how they operate, clear
expectations and interoperability requirements enable the site to fully
participate by exposing their existing services and infrastructure
through clear additional interfaces and standards. Similarly, focus on
interoperability promotes collaborating with other digital research
infrastructure providers.

\subsection*{Collaborative}
\addcontentsline{toc}{subsection}{Collaborative}%

The foundation for any successful truly federated organization must be
collaboration, not merely co-existence.  A federation, which incorporates
the breadth and diversity of researchers, provinces, funders and
personalities can only function if all parties come to the table in good
faith to discuss and negotiate. It can only be a success if the whole
becomes greater than the sum of its parts.

\begin{table}[ht] \centering \small {\sffamily
\rowcolors{2}{white}{gray!15}
\begin{tabular}{P{2.5in}|P{2.5in}}
\textcolor{cdaRed}{\textbf{Not Collaborative}} & \textcolor{cdaRed}{\textbf{Collaborative}} \\
\hline
\hline 
The focus is only on problems and challenges & The focus is on solutions and opportunities \\
Parties are focused on their local organizations & Parties are focused on the shared mission of meeting researcher needs \\
Parties are not willing to compromise & Parties are willing to give and take to most effectively achieve the shared mission \\
Coexisting silos & Whole greater than sum of its parts \\
\hline
\end{tabular}
}
\end{table}


This document outlines principles for a successful federated Compute
Canada, and one possible path to get there, but nothing is possible
without all parties wanting success and wanting to collaborate.

\proposal{The federation should aim to achieve more than the partners could achieve separately.}

Collaboration is not easy, and it often comes at the cost of taking more
time and energy. Working together, building consensus and getting people
onside requires time and compromise. And the only way this is possible
is if people are truly committed to success as a federation.

Collaboration cannot end at organizational borders. As very large-scale
research data and multi-institutional, multi-disciplinary consortia
become more and more important, close collaboration between and not just
within research support organizations will be vital. In the Prologue,
Shannon makes use of tools requiring compute, research data management,
and high performance networking.

\subsection*{Modern}
\addcontentsline{toc}{subsection}{Modern}%

A research service organization which uses technology to address
researcher needs must stay on top of new tools so that they can fully
meet those needs. Although researcher needs must always be the driver,
solutions change quickly, so the service organization must be building
experience to evaluate the benefits of these technologies if deployed on
a larger scale.

New tools can include hardware --- NVMe, FPGAs, and server-class ARM
CPUs are all technologies which could have significant impact on
research computing in the quite-near future --- but they can also be new
techniques for robustly and efficiently providing technical services.

\proposal{New training should continually be available for
emerging hardware and operational tools. }

An organization which embraces having modern tools must ensure there is
adequate staff time and training to learn and explore new hardware.
Small experimental systems must be made available to staff (and
interested researchers) to explore the suitability of new hardware for
research systems. Canada's early but measured adoption of GPUs took this
approach successfully. And such an organization ought not hesitate to
make use of commercial cloud providers when appropriate to make such new
technologies available.

\begin{table}[ht] \centering \small {\sffamily
\rowcolors{2}{white}{gray!15} \begin{tabular}{P{2.5in}|P{2.5in}}
\textcolor{cdaRed}{\textbf{Not Embracing Modern Tools}} &
\textcolor{cdaRed}{\textbf{Embracing Modern Tools}} \\ \hline \hline No
availability of experimental systems & Invests in new technology
for staff to explore for suitability for researcher use \\ Little
paid staff training & Provides staff with time and training in new
methods and techniques \\ 
Focus on 'tried-and-true' methods from supercomputing centres for
running systems and interacting with users & Focus on exploring,
customizing, and using approaches from across large-scale computing for
running systems, interacting with users. \\
Limited or no ongoing investigation of commercial service (ie: cloud):
providers are the competition & Commercial service providers are one of
many options for providing services to researchers\\
\hline
\end{tabular}
}
\end{table}

A modern organization also experiments with and trains on new
operational tools. As more and more companies rely on computer
infrastructure, the past decade and a half have led to improved
approaches to ensuring the services they provide are reliable and
effective. Techniques like Google's now widely adopted SRE
approach\footnote{\url{https://landing.google.com/sre/book.html}} or
Netflix's `Chaos Monkey' emphasize automation, rigorous testing, and
continuous improvement, allow staff to focus on providing higher quality
services.

\proposal{The federation should make use of best available tools
for interacting with, and supporting researchers.}

Since interactions with the researcher are so important, a modern
research support organization also takes advantage of new tools from
elsewhere for working with clients. Customer Relationship Management
(CRM) packages enable tracking researcher interactions and project
progress, allowing staff anywhere in Canada to come up to speed and
assist a remote researcher. In the Prologue, Shannon benefits from
up-to-date hardware, system methodologies, and interaction tools.

\section*{Governance Best Practices from Other Federations}
\addcontentsline{toc}{section}{Governance Best Practices from Other Federations}%
\addcontentsline{pro}{section}{Governance Best Practices from Other Federations}%
\stepcounter{section}

\nocite{*}
Managing and running a complex partnership like the one that is
responsible for our national platform, or any digital infrastructure
platform, may seem daunting. But it is vital to realize that federated
organizations are increasingly common in the nonprofit sector,
especially in Canada or amongst international NGOs, and that many
successful examples are available.


There is no one-size-fits-all approach to organizing a federation of
partners. However, significant thought and effort, in Canada and abroad,
has gone into examining governance and management models in a variety of
contexts; we can learn both from models that have worked very well, and
from cautionary tales. The authors have found the studies listed in the
References to be particularly valuable in informing this work.

Evidence of successful federations from across Canada and abroad suggest
that the choice of the basic architecture of our federation is sound.
But relationships and processes matter a great deal; \cite{widmer1999governance}
report that they ``...came to believe that federations were more
likely to be damaged by bad processes than bad structures''. Thus, we
focus on how several vital relationships can benefit from being
renegotiated in the light of what is done elsewhere. As a starting point
for discussion, we take the evidence of federations elsewhere and
propose steps for renewing the governance our federation.

\cite{mollenhauer_framework} pointed out that ``The goal of any federation should
be to get the benefits of a centralized structure, such as greater
efficiency and effectiveness'' --- and in our case, coherence --- ``
while retaining the benefits of local autonomy, such as community
responsiveness.''  It is fair to say that previous attempts at
organizing our federation have focused more on one or the other of those
sets of advantages. But armed with working examples from elsewhere in
Canada and abroad, we can aim to achieve a balance of both.

\subsection*{Clarity of Roles}
\addcontentsline{toc}{subsection}{Clarity of Roles}%

Indeed, evidence suggests some helpful moves have already been made.
\cite{mollenhauer_transformation} describes several successful Canadian organizations
going through a ground-up consolidation process very similar to our
history, with service providers organizing first into consortia, and
then into regional organizations. In the case of both the ALS Society of
Canada and the Parkinson Society of Canada, this move was made with the
intent to improve both the speed of national decision-making and the
coherence of local decision-making, while giving the federated
organization a healthy balance between national office and regional
offices.

\proposal{The federation partners should clearly delineate
the roles and responsibilities of the central office and the regional
organizations.}

However, in the case of our federation, this move may have been
incomplete. In both of those two cases part of the process involved
clear partnership agreements agreed to by the new regional organizations
and the central office as to the roles and responsibilities of each.
\cite{bigbrothers} briefly describe a similar process
with the American Cancer Society and the Girl Scouts of America --- in
the case of the Girl Scouts, the clarity of interactions offered by this
detailed description of roles allowed, for the first time, delivering
programs jointly with external partners

In a case study of the World Wildlife Fund US \citep{wwfus_casestudy},
staff described these sorts of agreements very
positively: ``We learned that we need these kinds of network initiatives
to be formal, not with bureaucracy, but with people needing to know each
other's roles.''

\subsection*{Boards}
\addcontentsline{toc}{subsection}{Boards}%

\proposal{The central office Board should be provided the
training and the support necessary to play their role in the federation.
}

\cite{mollenhauer_framework} offers a picture of a frequent challenge in Canadian
federated organizations which speaks to the central dilemma facing their
Boards:

\begin{quote}
``A clear distinction needs to be made between the role of the national
Board of Directors as it relates to the {[}central office{]} and its
role in the federation. Some national Board of Directors act as if they
have a greater ability to set direction and impose behaviors then is the
case. As a result, they undervalue the essential role of the {[}central
office{]} within the federation as convener and facilitator. \ldots{}
Even the language used by federations can be illustrative of the
confusion about the role of the {[}central office{]}. The {[}central
office{]} is a partner in the federation, but written and verbal
communication often describes the {[}central office{]} as the
federation.''
\end{quote}

This dilemma is particularly acute for a research support organization,
where the Board has responsibilities to both a national membership
needing national services, and a national funder requiring national
governance, but authority only over a central office --- and satisfying
their responsibilities requires the participation of all partners in the
federation. A Board in this situation can only be successful when their
responsibilities are aligned --- all federation partners are committed
to their shared mission, and national membership and funders that
understand the challenges but accept them for the sake of the benefits.

Even then, handling the conflicting roles of a central office board in a
federation is genuinely difficult, and we have asked board members to
date to take this on with little to no support on how best to proceed.
Most studies recommend board training that emphasizes the challenges and
possibilities of a federated system.

\proposal{The central office and regional Boards should
regularly meet to ensure alignment of governance. }

With the recognition that a central office Board does not operate in a
vacuum, some other possibilities for support present themselves. While
in our federation to date, there has been much effort in establishing
ongoing meetings between management and staff of the partners of our
federation, several works also suggest similar interactions between
members of the Boards of the partners to ensure alignment of governance,
not just management.

\subsection*{Membership}
\addcontentsline{toc}{subsection}{Membership}%

Evidence from the study of successful international advocacy NGOs \citep{Brown20121098},
suggest that membership in a federation should reflect the primary
accountability of the federation. In our case, this is to the
researchers, strongly suggesting that researchers or their
representatives should be members; we suggest that putting the burden on
researchers to govern the federation that should be working on their
behalf is unreasonable, and that the existing model of membership
comprising institutions in their role as representing researchers is a
reasonable compromise.

\proposal{Members should be given the access and support
they need to play an active role in the federation. }

Shared governance in a federation --- or indeed the governance of any
member-owned non-profit --- requires active participation of the members
to be successful. A finding of \cite{widmer1999governance} is that even
in federations where ``the membership may appear to have significant
powers, in practice, the influence of the membership may be limited by
infrequent opportunities to exercise power {[}...{]}, little control over
the agenda, lack of experience and cohesion among affiliate
representatives, and infrequent meetings of the membership'', whereas in
other organizations the membership is given many more opportunities to
participate in governance, from advisory roles to votes on policies.
Perhaps partly because of lack of visibility members have had into the
governance and management of the central office and the federation as a
whole, members to date have been reticent to fully participate. If our
project is to be successful, this needs to change; the federation has to
make sure its members have whatever support they need so that they can
take their full role within the federation.

In addition, the membership needs to be actively recruited to reflect as
broad a range of Canadian institutions as possible, and barriers to
membership should be reduced as much as feasible.

\subsection*{Federation-wide Decision Making}
\addcontentsline{toc}{subsection}{Federation-wide Decision Making}%

The role of a central office and how it complements the roles of the
other partners in the federation is crucial to a federation's success.
\cite{Brown20121098} identified several factors which
determine whether successful international organizations function as a
loosely-coupled network of allies or a more tightly-coupled federation.

In their work, they demonstrate that to the extent to which the work
being undertaken is long-term and coherence is needed, that federations,
being more tightly-coupled seem to work best. If the work is more short
term (as for individual short-term advocacy campaigns) or less coherence
is needed (as if each group was going to lobby only within its own
region), loosely-knit and perhaps even ad-hoc partnerships worked well.

We argue that Canadian research merits a long-term and coherent
computational platform for supporting research, meaning that a
federation, and not a loose network, is appropriate. But how should such
a federation operate internally? What should the roles of the individual
partners in the federation be, and how should decisions be made?

Management of a federation of co-equal partners can only be derived from
conensus.  Again from \cite{mollenhauer_framework}, a success factor in
federations is that:

\begin{quote}
``There is a clear understanding that leadership is shared across the
federation and there is acknowledgement of the role of consensus, not
authority, as key to decision-making. The CEO/Executive Director of
the national organization has strong skills in communication and
facilitation and puts high value on process as well as on delivering
results.''
\end{quote}

\proposal{Federation-wide decision making should be
supported by consensus.}

This doesn't mean that consensus must be achieved for every single
agenda item in a meeting --- that brings paralysis --- but on
decision-making processes themselves there must be explicit, formal
agreement, with clear distinctions between ``between decisions that need
unanimous or consensus agreement because they are critical (e.g. those
tied to risk management) versus those that need a majority (e.g. those
related to activities).'' And while a central office must be responsible
for those processes as the facilitator, which is a different role from
being the decision-maker.

The central office has played different roles over the years. As the
national arm of the platform, it will always be primarily responsible
for directly working with federal funders, national research
organizations and societies, and international partners.  Working with
those organizations gives the central office a different, birds-eye view
of the national research community.

These different perspectives matter: our federation's mandate is not
just to assist individual researchers already working with us but
Canadian researchers collectively. It is far too easy to focus too much
on either the forest or the trees, and the combination of hands-on and
birds-eye perspectives is vital in setting priorities, and consensus
decision making is required to bring these two perspectives together.

\subsection*{The Value of a Federation}
\addcontentsline{toc}{subsection}{The Value of a Federation}%

\cite{grossman2001managing} take an overview of five international
federations and look at what determines the relationship between the
partners. They point out that local autonomy and affiliation to a
central coherent framework aren't opposed; one can have partner
organizations with high autonomy and low (Outward Bound) or high (The
Nature Conservancy) affiliation and coherence. The determining factor in
the authors' view was the value of affiliation into a federation for the
partners; if there was high value in a federation, one would persist and
be stable, even in the presence of disagreements about operations or
strategy.

\proposal{The federation should make it clear internally, to
the research community, and to funders, the value of providing services
as a federation.}

In the case of our federation, there are several important ways a
federation can be valuable to the partners, although these have not yet
been fully realized. A federation can enable specialization, allowing
individual providers to focus their efforts on the services they are
best at providing, instead of trying to be all things to all researchers
in their jurisdiction; and it can allow the researchers in the
jurisdiction to access a wider range of services and expertise than
would otherwise be possible. However, those value propositions are
greatly diminished if the national platform focuses on uniformity rather
than interoperability.

The WWF-US case study mentioned earlier illustrates the importance of
need for value from working together for a federation. In the early
2000s, after years of WWF national offices being largely independent
with only certain aspects being set centrally, there was disagreement
about mission and priorities. This grew to tension between the central
office (WWF International, in Switzerland), and several national
offices, including WWF-US, the largest, which had seriously considered
leaving the network.

But in the mid 2000s a major international victory surrounding
conservation preservation in Tesso Nilo, Indonesia, had required
coordinated pressure from several national organizations and expertise
ranging from finance and marketing and the ecosystem science to the
local governance and land management practices; this collaboration,
which had grown organically and almost accidentally, convinced the
member organizations to restructure the federation around such projects
of global impact requiring global effort. Decisions are now largely made
through a ``network executive team'' involving the central office and
representatives for national and program offices, and local office
commitments to various programmes are spelled out in detailed documents
agreed to by both sides. While the national offices retain autonomy, the
network now acts in a much more coherent, integrated way; that increased
coherence has brought \emph{reduced} tension between central and local
offices due to the clarity of the mission.

On the other hand, if the value of federation isn't made clear, partners
may stop engaging with the federation or even depart, such as with the
recent situation with the Alzheimer's Association in the US \citep{alzheimers}.

\section*{Turning Principles into Operations}
\addcontentsline{toc}{section}{Turning Principles into Operations}%
\addcontentsline{pro}{section}{Turning Principles into Operations}%
\stepcounter{section}

Implementing governance best practices in our federation can greatly
improve how our federation functions, but we must make sure that the
federation operates in a way that lives up to the principles we choose
for our organization.

\proposal{Key Performance Indicators (KPIs) should reflect
our Principles.}

We must ensure that the metrics we set ourselves to measure success are
measuring the right things, rather than being easy to count.  This is
genuinely difficult; it is easier to measure interactions with
researchers than the number of interactions that no longer have to take
place because a task has been made simpler.  However, we must ensure that
our stated goals and our measurements of success coincide.

\subsection*{Services}
\addcontentsline{toc}{subsection}{Services}%

In the document we argue the best model for our federated organization
is services based. All partners in the federation provide services to
each other to build the national platform, and collectively to the
researchers. The Central Office is accountable to the researchers
(through the Members) and so the Central Office is responsible for
ensuring the Regions are accountable for meeting their agreed upon
services.

We propose that the priorities for the national platform will be set
through collaboration with the regions and the central office. These
priorities will be defined by a service or more appropriately a set of
services with the associated Service Level Agreements (SLAs) and
interfaces. The SLAs must have clearly defined metrics for evaluating
the effectiveness of the services provided.

\proposal{Services provided should come with agreed-upon
SLAs ensuring quality and interoperability. }

A federation colleague (\textit{i.e.}: a region or the central office) will 
propose taking on the responsibility for providing some or all of the
necessary service or services: allowing regions to build on existing
strengths or meet regional priorities. Ideally these services will be
assigned based on consensus, but for big ticket items where this is
unlikely to be possible by some pre-agreed upon process led by the
central office.

Once the agreement is made, the provider is free to implement the
service in any way they see fit, but are held accountable for meeting
the agreed-upon standards and metrics.

\proposal{Services should be piloted, with definitions of
success decided upon before the pilot. }

Most services should normally go through a pilot phase before being
provided more widely.  Deciding what success means for a pilot will
necessarily differ from service to service, and consensus should be
reached before the launching of a pilot what would merit a more
permanent, larger-scale roll-out.

It's worth noting that it may be perfectly reasonable for some regions
or institutions to provide services locally that are not part of the
national platform; there are some services which are not possible to
provide nationally, or there might not be sufficient demand for outside
of a given region, but one jurisdiction may be willing to fund
nonetheless. Not every type of research support necessarily has to be
shoehorned into a national platform framework.

\subsection*{Relationship between Federation Partners}
\addcontentsline{toc}{subsection}{Relationship between Federation Partners}%

While federations of equal partners must have a basis in consensus, each
partner has specific roles to play.

\proposal{The central office should be responsible for
nation-wide needs assessments and researcher satisfaction. }

The central office must be the convener of the ``birds-eye view'' of the
Canadian research community, working closely with federal funding
agencies, national scholarly communities, and other research service
providers who can identify gaps in the research ecosystem, or
underserved communities.  In that role, the central office should be
responsible for performing nation-wide needs assessments, measuring
researcher satisfaction, and ensuring the input of national researcher
communities into the federation's discussions.

\proposal{The agenda should be managed by the central
office, and consensus should be found or built around priorities. }

As a convener and facilitator, and as the organization developing
nation-wide assessments, the central office will always be the setter of
the federation's agenda.  It will need to be the body that drives the
push for evidence-based consensus around national priorities, planning
next steps, and where necessarily, building partnerships outside of the
federation to accomplish the federation's agreed-upon goals.

\proposal{The central office should be responsible for
monitoring and enforcing interoperability and other SLAs on the
platform.}

Coherence of the national platform, and adherence to interoperability
and other agreed-upon standards, will necessarily be the responsibility
of the central office. It is this body that will perform monitoring of
these service levels, and testing interoperability. It is also the
central office's responsibility to ensure that there are accountability
measures in place for service providers that are not meeting their SLAs.
However, since failure to provide interoperability or service levels is
a failure felt by the entire platform, not just the central office,
other federation partners must also play a role in enforcing these
standards.

The needs assessment and SLA or interoperability monitoring roles are
vital, and generally will be the primary technical roles of the central
office, as researcher-facing technical services and operations will
generally be best managed by the regional organizations and sites.

\proposal{Responsibility for operations of researcher-facing
services should generally belong to the regional organizations.}

Being nimble, and being able to quickly tell if a researcher-facing
service is successful or if it should be changed, will normally require
researcher-facing services to be provided organizationally close to the
researchers. This will generally mean that such services will be housed
in one or more regional organization.

\proposal{Internal federation services can be provided by
any partner, or externally. }

On the other hand, internal services necessary for the administration
and operation of the federation itself --- CRMs, email, dashboards and
monitoring, finance services --- might reasonably be housed in any of a
number of places; any federation partner might propose hosting such a
service.  As with a researcher-facing service, such a service would come
with an SLA.  Externally-provided services would still require one
federation partner to be responsible for the SLA and interacting with
the vendor.

\subsection*{Relationship with Other Partners}
\addcontentsline{toc}{subsection}{Relationship with Other Partners}%

As the uses of computation and data broaden, and become more integrated
into all areas of scholarship, investigators will increasingly need
services that require coordination of remote and institutional
computation and storage, networking, data management, and other research
services. It won't --- and shouldn't --- matter to those researchers how
this coordination happens; either across multiple organizational
boundaries or within a single organization so long as the access to
resources is seamless.

\proposal{The federation must work closely with other
digital infrastructure providers and research services organizations in
service design, service delivery, and future planning.}

Several models for how this close collaboration could work have been
proposed, and should be discussed by the community at large. As
suggested above, however, structures matter less in and of themselves
than they do for their effect on processes; and it is the process that
is crucial here. Just as it is unacceptable for researchers to have to
routinely cobble together resources to support their project within a
researcher-centred organization, it is unacceptable for researchers to
have to manage for themselves the more complex task of coordinating
resources across research support organizations.

In research, collaboration means much more than participants announcing
to each other what they have done or what they intend to do; just so
with research service organizations. A meaningful collaboration, one
that can make the best use of each other's strengths and resources,
means frequent discussions through planning, implementation, and
execution phases of a project.

\section*{Conclusion and Next Steps}
\stepcounter{section}
\addcontentsline{toc}{section}{Conclusion and Next Steps}%

The purpose of this document is not to advocate in particular for the
proposals contained within (although the proposals made here reflect
genuinely-held convictions, rather than being straw-man arguments). The
purpose is to start in earnest a conversation that is overdue, allowing
the community to come to a consensus about what the internal
organizational structure of Compute Canada should be, how it should make
decisions, and how it should offer services to Canadian researchers and
scholars.

The most important next step, then, is for you to have this discussion
with colleagues locally and across the country, disagreeing vehemently
initially on some points, and coming to agreement on others. 
Our document focuses on the organization that is Compute Canada.
However, as mentioned the principles and proposals presented can be
applied to any digital infrastructure organization. Furthermore, the
ultimate organization or governance structure that supports the delivery
of research computing support could be any number of a wide range of
models. Open discussions about that model, or various options could be a
valuable step forward. However, we advocate that regardless of the model
it is critical that researcher needs be the first and most important
consideration.

The members and regions can build a successful and coherent national
platform that works the way the community wants it to, but they cannot
do so before the community tells them what destination they should aim
for. The Canadian research and research computing communities can do
great things together. Let's get started.

\newpage

\bibliographystyle{chicago-annote}
\addcontentsline{toc}{section}{References}%
\bibliography{refs}

\clearpage

\makebackcover
\end{document}
